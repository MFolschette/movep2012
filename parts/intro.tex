\section{Introduction}
As regulatory phenomena play a crucial role in biological systems, they need to be studied accurately.
Biological Regulatory Networks (BRNs) consist in sets of either positive or negative mutual effects between the components.
Besides continuous models of physicists, often designed through systems of ordinary
differential equations, a discrete modeling approach was initiated by René Thomas in 1973
\cite{Thomas73} allowing the representation of the different levels of a component, such as concentration or expression levels, as integer values.
%Qualitative state graphs may be derived from which we are able to formally find out all the possible behaviors.
Nevertheless, these dynamics can be precisely established only with regard to some kind of ``focal points'', related to as Thomas' parameters, indicating the evolutionary tendency of each component.

Thomas' modeling has motivated numerous works around the link between the influences
%(summarized in the Interaction Graph)
and the possible dynamics (e.g., \cite{RiCo07}), % RRT08
model reduction (e.g., \cite{Naldi09}), %formal checking of dynamics (e.g., \cite{Richard06,Naldi07}), 
or the incorporation of time (e.g., \cite{Siebert06,Ahmad08}) % Ahmad08
%and probability (e.g., \cite{Twardziok10-CMSB}) dimensions,
to name but a few.
Other approaches related to our work, which rely on on temporal logic~\cite{Khalis09} and constraint programming~\cite{20646302,DBLP:conf/ipcat/CorblinFTCT12},
aim at determining models consistent with partial data on the regulatory structure and dynamics.
%Our work is also related to the approach of \cite{Khalis09} which relies on temporal logic, and \cite{20646302,DBLP:conf/ipcat/CorblinFTCT12} which uses constraint programming.
%Both aim at determining a class of models which are consistent with available partial data on the regulatory structure and dynamical properties.
While the formal checking of dynamical properties is often limited to small networks because of the
state graph explosion, the main drawback of this framework is the difficulty to specify Thomas'
parameters, especially for large networks.
In our approach, we intend to focus on the Thomas' parameters inference.

In order to address the formal checking of dynamical properties within very large BRNs, we recently
introduced in \cite{PMR10-TCSB} a new formalism, named the \emph{``Process Hitting''} (PH), to model
concurrent systems having components with a few qualitative levels.
A PH describes, in an atomic manner, the possible evolutions of a “process” (representing one
component at one level) triggered by the hit of at most one other “process” in the system.
%A PH describes, in an atomic manner, the possible evolution of the level of a component triggered by at most one other component.
This particular structure makes the formal analysis of BRNs with hundreds of components tractable \cite{PMR12-MSCS}.
A PH model can be built based on information found in the literature about the local influences between components.
It is then suitable, according to the precision of this information, to model BRNs with different levels of abstraction by capturing the most general dynamics.

In this work\footnote{The details of our method are presented in \cite{FPIMR12-CMSB}.}, we show that starting from one PH model, it is possible to find the underlying interactions.
We perform an exhaustive search for the possible interactions on one component from all the
others, consistently with the knowledge of the dynamics expressed in PH.
The second phase of our work concerns the Thomas' parameters inference.
It consists in abducing the (possibly large) nesting set of parameters which, together with other given conditions, sufficiently derives satisfaction of the known cooperating constraints.
The resulting dynamics are ensured to respect the PH dynamics, \ie no spurious transitions are
made possible.

%The first benefit of this work is that such an approach
The first benefit of our approach is that it makes possible the construction refining of BRNs with a partial and progressively brought knowledge in PH, while being able to export such models in the Thomas' framework.
Our second contribution is to enhance the knowledge of the formal links between both modelings.
As BRNs are not limited to Boolean values, the whole method can be applied to multi-valued models;
furthermore, the method can be applied to large BRNs (up to 40 components).

\paragraph{Outline.}
\pref{sec:frameworks} recalls the PH and Thomas frameworks;
\pref{sec:infer-IG} defines the IG inference from PH;
\pref{sec:infer-K} details the enumeration of Thomas parametrizations compatible with a PH;
\pref{sec:examples} gives some information about the implementation of the method.
