\section{Introduction}
As regulatory phenomena play a crucial role in biological systems, they need to be studied accurately.
Biological Regulatory Networks (BRNs) consist in sets of either positive or negative mutual effects between the components.
Besides continuous models of physicists, often designed through systems of ordinary
differential equations, a discrete modeling approach was initiated by René Thomas in 1973
\cite{Thomas73} allowing the representation of the different levels of a component, such as concentration or expression levels, as integer values.
Nevertheless, these dynamics can be precisely established only with regard to some kind of ``focal points'', related to as Thomas' parameters, indicating the evolutionary tendency of each component.
This modeling has motivated numerous works (e.g., \cite{RiCo07,Naldi09,Siebert06,Ahmad08}),
and other approaches related to our work, which rely on temporal logic~\cite{Khalis09} and constraint programming~\cite{20646302,DBLP:conf/ipcat/CorblinFTCT12},
aim at determining models consistent with partial data on the regulatory structure and dynamics.
While the formal checking of dynamical properties is often limited to small networks because of the
state graph explosion, the main drawback of this framework is the difficulty to specify Thomas'
parameters, especially for large networks.

In order to address the formal checking of dynamical properties within very large BRNs, we recently
introduced in \cite{PMR10-TCSB} a new formalism, named the \emph{``Process Hitting''} (PH), to model
concurrent systems having components with a few qualitative levels.
A PH describes, in an atomic manner, the possible evolutions of a “process” (representing one
component at one level) triggered by the hit of at most one other “process” in the system.
This particular structure makes the formal analysis of BRNs with hundreds of components tractable \cite{PMR12-MSCS}.
PH is suitable, according to the precision of this information, to model BRNs with different levels of abstraction by capturing the most general dynamics.

In this work\footnote{The formal details of our method are presented in \cite{FPIMR12-CMSB}.}, we show that starting from one PH model,
it is possible to find the underlying interactions, then the underlying Thomas' parameters.
It relies on an exhaustive search of the interactions between components of the PH model,
and an enumeration of the (possibly large) nesting set of valid parameters,
so that the resulting dynamics are ensured to respect the PH dynamics, \ie no spurious transitions are made possible.

The first benefit of our approach is that it makes possible the construction refining of BRNs with a partial and progressively brought knowledge in PH, while being able to export such models in the Thomas' framework.
Our second contribution is to enhance the knowledge of the formal links between both modelings.
The method can be applied to large BRNs (up to 40 components).
