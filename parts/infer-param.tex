\section{Parametrization inference}\label{sec:infer-K}

Given the IG inferred from a PH as presented in the previous section, one can find the discrete parameters that model the behavior of the studied PH using the method presented in the following.
As some parameters may remain undetermined, another step allows to enumerate all parametrizations compatible with the inferred parameters.

\subsection{Independent parameters inference}

This subsection presents some results related to the inference of independent discrete parameters from a given PH,
equivalent to those presented in \cite{PMR10-TCSB}.
We suppose in the following that the considered PH is well-formed for parameters inference: its inferred IG does not contain any unsigned edge,
and in each sort, all processes activating (resp. inhibiting) another component share the same behavior.
Let $K_{a,A,B}$ be the parameter we want to infer for a given component $a \in \Gamma$
and $A \subset \GRNreg{a}$ (resp. $B \subset \GRNreg{a}$) a set of its activators (resp. inhibitors).
This inference, as for the IG inference, relies on the search of focal processes of the component for the given configuration of its regulators.

For each sort $b \in \GRNreg{a}$, we define a context that contains all processes of $b$ activating (resp. inhibiting) $a$ if $b \in A$ (resp. $B$).
From all contexts of all predecessors of $a$, we create a global context that represents the configuration $A,B$ (including the cooperative sorts involved).
The parameter $K_{a,A,B}$ specifies towards which values $a$ eventually evolves as long as this context holds, which is precisely given by the set of focal processes.

\begin{example*}
In the PH of \pref{fig:runningPH}, we have in particular:
\begin{align*}
\focals(b,L_b,\{a_0, a_1\}) = \{b_0, b_1\} &\text{, which gives: } K_{b,\{a\},\emptyset} = [0 ; 1], \\
\focals(a,L_a,\{b_1, c_1, bc_{11}\}) = \{a_2\} &\text{, which gives: } K_{a,\{b,c\},\emptyset} = [2 ; 2] \text{, and} \\
\focals(a,L_a,\{b_1, c_0, bc_{10}\}) = \{a_1\} &\text{, which gives: } K_{a,\{b\},\{c\}} = [1;1].
\end{align*}
\end{example*}

This method sometimes faces cases with opposite effects on a component, leading to either an indeterministic evolution or to oscillations.
Such an indeterminism is not possible in a BRN, and the inference of the targeted parameter is impossible.
In order to avoid such inconclusive cases, one has to ensure that no such behavior is allowed
by either removing undesired actions or using cooperative sorts to prevent opposite influences between regulators.

\subsection{Abductive reasoning to find admissible parametrizations}\label{ssec:admissible-K}

In the following, we try to constrain all parameters that are left undetermined with the method presented in the previous subsection.
We consider that a parameter is valid if any transition it involves in the resulting BRN is allowed by the studied PH by actions that represent this behavior.
We also add some biological constraints on the whole parametrizations, given in \cite{BernotSemBRN}.
These constraints lead to a family of admissible parametrizations which we can enumerate and are ensured to observe a coherent behavior that is included in the original PH.

This approach can be considered as abductive reasoning as some information is added by the enumeration.
If we denote:
\begin{itemize}
  \item $M$ the fact that the behavior of the resulting BRN observes the dynamics of the PH,
  \item $B$ the fact (which is granted) that the IG and the series of necessary parameters inferred from the PH are parts of the resulting BRN,
  \item $H_K$ the hypothesis that $K$ is an admissible complete parametrization,
\end{itemize}
then the parametrizations $K$ that answer our expectations are the ones so that:
\begin{itemize}
  \item $H_K$ is compatible with $B$, that is, all parameters of $K$ are compatible with the inferred parameters,
  \item $B \wedge H_K \models M$, that is, the inferred IG together with $K$ represent a BRN observing the behavior included into the dynamics of the original PH.
\end{itemize}

Answer Set Programming (ASP) \cite{Baral03} turns out to be effective for the enumerative searches developed in this paper,
as it efficiently tackles the inherent complexity of the models we use, thus allowing an efficient execution of the formal tools developed.
Furthermore, ASP finds a particularly interesting application in the research of admissible parametrizations regarding the properties presented above, as this enumeration can be naturally formulated with the use of aggregates, and constraints allow to remove all non-admissible models.
